\documentclass[tikz]{standalone}
\usepackage{luatex85}
\usepackage{tikz}
\usetikzlibrary{arrows}
\usetikzlibrary{graphs}
\usetikzlibrary{graphdrawing}
\usetikzlibrary{shapes}
\usegdlibrary{layered}

% Define the set of semi automated chart generation tools
% Sets the standard shapes used in the chart to the colour requested
\newcommand{\generateBlockStyles}[1]{
    \tikzstyle{function} = 
        [rectangle, draw, fill=#1!50, rounded corners, 
        text centered, text width = 5em, minimum height=4em]
    \tikzstyle{condition} = 
        [diamond, draw, fill=#1!10, 
        text centered, text width = 5em, minimum height=4em]
    \tikzstyle{operation} = 
        [rectangle, draw, fill=#1!20, rounded corners, 
        text centered, text width = 5em, minimum height=4em]
    \tikzstyle{parameters} = 
        [draw, ellipse,fill=black!20, 
        text centered, text width = 5em, minimum height=4em]
    \tikzstyle{loop} = [draw, ellipse, fill=#1!30, 
        text centered, text width = 5em, minimum height=4em]
    \tikzstyle{callFuncStyle} = 
        [rectangle, draw, fill=#1!20, rounded corners, 
        text centered, text width = 5em, minimum height=4em]
    \tikzstyle{blank} = []
}
% Generate the function declaration and parameters input
% Usage \funcDec{#}{Name}{Parameter List}
\newcommand{\funcDec}[3]{
    \node [function] (#1) {#2};
    \node [parameters] (parameters#1) {#3};
    \draw (parameters#1) edge[->] (#1);
}
% Conclusion Generator
% Usage \funcConc{#}{Return Value(s)}{Previous Node}
\newcommand{\funcConc}[3]{
    \node [function] (#1) {return #2};
    \draw (#3) edge[->] (#1);
}
% Variable Generator
% Usage \varDec{#}{Variables}{Previous Node}
\newcommand{\varDec}[3]{
    \node [operation] (#1) {initiate variable(s): #2};
    \draw (#3) edge[->] (#1);
}
% Generic Condition Generator
% Usage \genericCond{#}{Condition}{Previous Node}
\newcommand{\genericCond}[3]{
    \node[condition] (#1) {#2};
    \draw (#3) edge[->] (#1);
}
% While Loop Generator
% Usage \whileDec{#}{Condition}{Previous Node}
\newcommand{\whileDec}[3]{
    \node[loop] (#1) {#2};
    \draw (#3) edge[->] (#1);
}
% Generic Operation Generator
% Usage \operationDec{#}{Data}{Previous Node}
\newcommand{\operationDec}[3]{
    \node[operation] (#1) {#2};
    \draw (#3) edge[->] (#1);
}
% Call Upon Other Functions Generator
% Usage \operationDec{#}{Name}{Parameters}{What is done with the return}{Previous Node}
\newcommand{\callFunc}[6]{
    \node[callFuncStyle] (callFunc#1) {#2};
    \node [parameters] (funcParameters#1) {#3};
    \node[operation] (#1) {which returns: #4};
    \draw (funcParameters#1)  edge[->] (callFunc#1);
    \draw (#5) edge[->] (callFunc#1);
    \draw (callFunc#1) edge[->] (#1);
}
% add special edges
% Usage \operationDec{from}{to}
\newcommand{\edgeDraw}[2]{
    \draw (#1) edge[->] (#2);
}

\begin{document}
\generateBlockStyles{darkgray}
\begin{tikzpicture}[node distance = 1cm and 1cm]
    \begin{scope}[layered layout, sibling distance=25mm, every edge/.style={draw=black}]
        % This is a main function. It's called final\_cal1
        \funcDec{0}{final\_cal1}{'int', 'in'}
        % This is a return statement. It's id is 1
        \funcConc{1}{ in  -- }{0}
        % This is an error line
        \operationDec{2}{Error! Type Mismatch!}{1}
        % This is a code line
        \operationDec{3}{End of Function}{1}

    \end{scope}
\end{tikzpicture}
\generateBlockStyles{red}
\begin{tikzpicture}[node distance = 1cm and 1cm]
    \begin{scope}[layered layout, sibling distance=25mm, every edge/.style={draw=black}]
        % This is a main function. It's called called
        \funcDec{4}{called}{'int', 'in'}
        % This is a return statement. It's id is 5
        \funcConc{5}{ final\_cal1  functionCall  in }{4}
        % This is an error line
        \operationDec{6}{Error! Type Mismatch!}{5}
        % This is a code line
        \operationDec{7}{End of Function}{5}

    \end{scope}
\end{tikzpicture}
\generateBlockStyles{magenta}
\begin{tikzpicture}[node distance = 1cm and 1cm]
    \begin{scope}[layered layout, sibling distance=25mm, every edge/.style={draw=black}]
        % This is a main function. It's called ca1ls
        \funcDec{8}{ca1ls}{'float'}
        % This is a code line
        \operationDec{9}{ int  i  =  42 }{8}
        % This is a code line
        \operationDec{10}{ float  f  =  23.000000 }{9}
        % This is an if statement. It's id is 11
        \genericCond{11}{Condition}{10}
        \operationDec{11I}{If  called  functionCall  i  ==  41 }{11}
        % This is an error line
        \operationDec{12}{Error! Type Mismatch!}{11}
        % This is a code line
        \operationDec{16}{ f  =  1.000000 }{11I}
        % This is an else statement. It's id is 14
        \operationDec{14}{Otherwise}{11I}
        % This is a code line
        \operationDec{15}{ f  =  0.000000 }{14}
        % This is a return statement. It's id is 17
        \funcConc{17}{ f }{11}
        % This is a code line
        \operationDec{18}{End of Function}{17}
        % This is an edge. It connects end nodes.
        \edgeDraw{15}{17}
        % This is an edge. It connects end nodes.
        \edgeDraw{16}{17}

    \end{scope}
\end{tikzpicture}
\end{document}